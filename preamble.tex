% ------------------------------------------------------------------
%     LaTeX Preamble -- Personal documents of Zilong Liang
% ------------------------------------------------------------------

% --- Document Class -----------------------------------------------
\documentclass[UTF-8]{article}  % English
% \documentclass[UTF-8]{ctexart}  % Chinese
% ------------------------------------------------------------------

% --- Packages -----------------------------------------------------
\usepackage{geometry}  % Geometry
\usepackage{fancyhdr}  % Page-head and page-foot
\usepackage[bookmarksnumbered=true]{hyperref}  % Hyper-reference
\usepackage{makeidx}  % Index
% \usepackage{fontspec,xeCJK}  % XeLaTeX Font setting
\usepackage{pifont}  % Circled number
\usepackage[perpage]{footmisc}  % Initialize footnote number every page
\usepackage{setspace}  % Line space
\usepackage{graphicx}  % Graphics
\usepackage{tabularx,booktabs,threeparttable}  % Enhanced tabular
\usepackage{caption,subcaption}  % Caption 
\usepackage{xcolor,tcolorbox}  % Color and colored text box
\usepackage[thmmarks]{ntheorem}  % Theorem environment
\usepackage{amsmath,amssymb,bm}  % AMS math symbols
\usepackage{cases}  % Cases for math
\usepackage{xfrac}  % Fraction
\usepackage{units}  % Units
\usepackage{listings}  % Code environment
\usepackage{algorithmicx,algpseudocode}  % Pseudocode
\usepackage{tikz}  % TikZ
  \usetikzlibrary{positioning}  % Relative position 
  \usetikzlibrary{cd}  % Change diagram
  \usetikzlibrary{automata}  % Automata
  \usetikzlibrary{backgrounds}
  \usetikzlibrary{fit}
  % \usetikzlibrary{external}  % External cache
% ------------------------------------------------------------------

% --- Settings -----------------------------------------------------
% --- Geometry
\geometry{a4paper,centering,scale=0.75}
% --- Page style (page head and page foot)
\pagestyle{plain}
% --- Vertical spaces for float object (for Chinese articles)
% \renewcommand\arraystretch{1.3}
% --- Bibliography format
\bibliographystyle{siamplain}  % Using SIAM style
% --- Footnote format
% \renewcommand\thefootnote{\ding{\numexpr171+\value{footnote}}}
% --- Connect theorem environemts with sections
% \numberwithin{equation}{section}
% \numberwithin{figure}{section}
% \numberwithin{table}{section}
% ------------------------------------------------------------------

% --- New Environments ---------------------------------------------
% --- Inline quote
\newcommand{\ilquote}[1]{{\slshape {#1}}}
% --- Short quote
\newenvironment{pquote}{\begin{quote} \slshape}{\end{quote}}
% --- Block quote
\newenvironment{bquote}{\begin{quotation} \slshape}{\end{quotation}}
% --- Emphasize
\renewcommand{\emph}[1]{{\textbf{#1}}}
% ------------------------------------------------------------------

% --- New Math Commands and Environments ---------------------------
% --- Symbols
\newcommand{\diag}{\mathrm{diag}}  % Diagonal matrix
\newcommand{\trans}{\mathrm{T}}  % Transpose
\newcommand{\mo}[1]{\left| {#1} \right|}  % Mod
\newcommand{\norm}[1]{\left\| {#1} \right\|}  % Norm
\newcommand{\numset}{\mathbb}  % Number sets
\newcommand{\conj}{\overline}  % Conjugate
\newcommand{\all}{\forall \,}  % For all
\newcommand{\exi}{\exists \,}  % Exists
\newcommand{\me}{\mathrm{e}}  % e
\newcommand{\mi}{\mathrm{i}}  % i
\newcommand{\st}{\, \mathrm{s.t.} \,}  % "So that"
\newcommand{\diff}{\mathop{}\!\mathrm{d}}  % Differentiation
\newcommand{\pdiff}{\mathop{}\!\partial}  % Partial diff
\DeclareMathOperator*{\argmax}{argmax}  % argmax
\DeclareMathOperator*{\argmin}{argmin}  % argmin
% --- Theorem environments
{
  \theorembodyfont{\itshape}
  \newtheorem{definition}{\textsc{Definition}}[section]  % Definition
  \newtheorem{theorem}{\textsc{Theorem}}[section]  % Theorem
  \newtheorem{proposition}{\textsc{Proposition}}[section]  % Proposition
  \newtheorem{corollary}{\textsc{Corollary}}[section]  % Corollary
  \newtheorem{lemma}{\textsc{Lemma}}[section]  % Lemma
  \newtheorem{example}{\textsc{Example}}[section]  % Example
  \newtheorem{alg}{\textsc{Algorithm}}[section]  % Algorithm
  {\theoremstyle{nonumberplain} \theoremsymbol{\mbox{$\Box$}}
  \newtheorem{proof}{\textit{Proof.}}}  % Proof
}
% ------------------------------------------------------------------

% --- TikZ Settings ------------------------------------------------
% --- Cache
% \tikzexternalize[prefix=TikZ-Temp/]

% --- Global settings
\tikzset{shorten >=1pt,node distance=3cm,on grid,auto,
         every state/.style={draw=none,fill=black!15},
         accepting/.style={draw=black!60,very thick,fill=black!15},
         every node/.style={font=\small}, bend angle=60}
% ------------------------------------------------------------------

% --- Engineering Settings -----------------------------------------
% --- Code
\lstset{
  columns=fullflexible, % 调整字距
  basicstyle=\ttfamily, % 调整主字体
  keepspaces=true, % 保持原代码空格数
  showstringspaces=false, % 不显示原代码空格占位符
  numbers=left, % 显示行号
  numberstyle=\footnotesize\ttfamily\color{gray}, % 调整行号格式
  frame=none, % 不显示边框
  commentstyle=\ttfamily\color{gray}, % 调整注释格式
  defaultdialect=[LaTeX]TeX % 使用 LaTeX 为 TeX 默认方言
}
% Pseudocode
\algrenewcommand{\algorithmiccomment}[1]{\hskip3em%
  \textcolor{gray}{\ttfamily\slshape // #1}}
\algnewcommand{\LComment}[1]{\textcolor{gray}%
  {\ttfamily\slshape // #1}}
\algrenewtext{For}[2]{\algorithmicfor\ #1 \textbf{to}\ #2%
  \algorithmicdo}
\algnewcommand{\Returns}{\textbf{returns}\ }
\newcommand{\hy}{\text{-}}
% ------------------------------------------------------------------

% --- New Engineering Environments ---------------------------------
% --- Codeblock (Usage: \codeblock[language=lang]{filename})
\newcommand{\codeblock}[2][]{\begin{spacing}{1.0}%
  \lstinputlisting[#1]{#2}\end{spacing}}
% --- Pseudocode (pcode)
\newenvironment{pcode}{\smallskip\begin{spacing}{1.1}\begin{algorithmic}%
  \upshape}{\end{algorithmic}\end{spacing}}
% ------------------------------------------------------------------
