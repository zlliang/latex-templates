% ------------------------------------------------------------------
%     LaTeX Preamble -- Presentation Documents of Zilong Liang
% ------------------------------------------------------------------

% --- Document Class -----------------------------------------------
\documentclass{beamer}  % English
% \documentclass[9pt]{ctexbeamer}  % Chinese
% ------------------------------------------------------------------

% --- Packages -----------------------------------------------------
\usepackage{amsmath,amssymb,bm}  % AMS math symbols
\usepackage{cases}  % Cases for math
\usepackage{units}  % Units
\usepackage{setspace}  % Line space
\usepackage{graphicx}  % Graphics
\usepackage{caption,subcaption}  % Caption 
\usepackage{tabularx,booktabs,threeparttable}  % Enhanced tabular
% ------------------------------------------------------------------

% --- Beamer Settings ----------------------------------------------
\usetheme{default}
\usecolortheme{spruce}
\usefonttheme{professionalfonts}
\bibliographystyle{apalike}
\setbeamertemplate{navigation symbols}{}
% ------------------------------------------------------------------

% --- New Math Symbols --------------------------------------------
\DeclareMathOperator{\diag}{diag}  % Diagonal matrix
\DeclareMathOperator{\mspan}{span}  % Spanning space
\DeclareMathOperator{\rank}{rank}  % Rank
\renewcommand{\t}{\mathrm{T}}  % Transpose
\newcommand{\h}{\mathrm{*}}  % Conjugate transpose
\newcommand{\mo}[1]{\left|{#1}\right|}  % Mod
\newcommand{\norm}[1]{\left\|{#1}\right\|}  % Norm
\newcommand{\numset}{\mathbb}  % Number sets
\newcommand{\conj}{\overline}  % Conjugate
\newcommand{\all}{\forall\,}  % For all
\newcommand{\exi}{\exists\,}  % Exists
\newcommand{\e}{\mathrm{e}}  % e
\renewcommand{\i}{\mathrm{i}}  % i
\newcommand{\st}{\,\mathrm{s.t.}\,}  % "Such that" or "Subject to"
\renewcommand{\d}{\mathop{}\!\mathrm{d}}  % Differentiation
\newcommand{\pd}{\mathop{}\!\partial}  % Partial diff
\renewcommand{\hat}{\widehat}  % Wide hat
\renewcommand{\tilde}{\widetilde}  % Wide tilde
\DeclareMathOperator*{\argmax}{argmax}  % argmax
\DeclareMathOperator*{\argmin}{argmin}  % argmin
% ------------------------------------------------------------------
