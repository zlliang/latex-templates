% ------------------------------------------------------------------
%     LaTeX Preamble -- Presentation Documents of Zilong Liang
% ------------------------------------------------------------------

% --- Document Class -----------------------------------------------
\documentclass{beamer}  % English
% \documentclass[fontset=mac]{ctexbeamer}  % Chinese
% ------------------------------------------------------------------

% --- Packages -----------------------------------------------------
\usepackage{amsmath,amssymb,bm}  % AMS math symbols
\usepackage{cases,units}  % Cases and units for math
\usepackage{setspace}  % Line space
\usepackage{graphicx}  % Graphics
\usepackage{caption,subcaption}  % Caption 
\usepackage{tabularx,booktabs,threeparttable}  % Enhanced tabular
% \usepackage{algorithm,algorithmicx,algpseudocode}  % Pesudocode
% \usepackage{tikz,pgfplots}  % TikZ and PGFPlots
% \usetikzlibrary{positioning}  % Relative position 
% \usetikzlibrary{cd}  % Change diagram
% \usetikzlibrary{backgrounds}  % Backgrounds
% \usetikzlibrary{fit}  % Fit
% \usetikzlibrary{intersections}  % Fit
% \pgfplotsset{compat=1.16}
% ------------------------------------------------------------------

% --- Beamer Settings ----------------------------------------------
\useoutertheme{infolines}
\usefonttheme[onlymath]{serif}
\usecolortheme[RGB={0,92,175}]{structure}
\usecolortheme{whale}\usecolortheme{orchid}
\setbeamertemplate{navigation symbols}{}
% --- Show table of contents at section beginnings
\AtBeginSection{
  \begin{frame}{Outline}
    \tableofcontents[currentsection]
  \end{frame}
}
% ------------------------------------------------------------------

% --- New Math Symbols ---------------------------------------------
\DeclareMathOperator{\diag}{diag}  % Diagonal matrix
\DeclareMathOperator{\tr}{tr}  % Trace
\DeclareMathOperator{\rank}{rank}  % Rank
\DeclareMathOperator{\Hess}{Hess}  % Hessian
\DeclareMathOperator{\Prob}{P}  % Probability
\DeclareMathOperator{\E}{E}  % Expectation
\DeclareMathOperator{\cov}{cov}  % Covariance
\DeclareMathOperator{\grad}{grad}  % Gradient
\DeclareMathOperator{\sgn}{sgn}  % Sign function
\DeclareMathOperator*{\argmax}{argmax}  % argmax
\DeclareMathOperator*{\argmin}{argmin}  % argmin
\def\l{\left}  % Left
\def\r{\right}  % Right
\def\t{\mathrm{T}}  % Transpose
\def\h{\mathrm{*}}  % Conjugate transpose
\def\abs#1{\l|{#1}\r|}  % Mod
\def\norm#1{\l\|{#1}\r\|}  % Norm
\def\numset{\mathbb}  % Number sets
\def\R{\numset{R}}  % Real number set
\def\C{\numset{C}}  % Complex number set
\def\conj{\overline}  % Conjugate
\def\e{\mathrm{e}}  % e
\def\i{\mathrm{i}}  % i
\def\d{\mathop{}\!\mathrm{d}}  % Differentiation
\def\pd{\mathop{}\!\partial}  % Partial diff
\def\hat#1{\widehat{#1}}  % Wide hat
\def\tilde#1{\widetilde{#1}}  % Wide tilde
\def\geq{\geqslant}  % Greater than
\def\leq{\leqslant}  % Less than
\def\qed{\hfill\ensuremath{\square}}  % End of proof
\def\st{\,\mathrm{s.t.}\,}  % "Such that" or "Subject to"
% ------------------------------------------------------------------

% --- Algorithm ----------------------------------------------------
% \renewcommand{\algorithmicrequire}{\textbf{Input:}}
% \renewcommand{\algorithmicensure}{\textbf{Output:}}
% ------------------------------------------------------------------
